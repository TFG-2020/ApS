\documentclass{article}

\usepackage{graphicx}
\usepackage[dvipsnames]{color}
\usepackage[hidelinks]{hyperref}
\usepackage[numbers]{natbib}
%pPara poder modificar los margenes
\usepackage{vmargin}
%Para usar el español
\usepackage[spanish]{babel}
\usepackage[utf8]{inputenc}
\begin{document}
%Portada
\setpapersize{A4}
\begin{titlepage}
	\centering
	{\includegraphics[width=0.8\textwidth]{logo}\par}
	\vspace{1cm}
	{\Large Facultad de Ingeniería Informática \par}
	\vspace{3cm}
	{\scshape\Huge Aplicación web de soporte al Aprendizaje-Servicio \par}
	\vspace{5cm}
	{\textbf\Large Autores \par}
	{\Large Daniela Nicoleta \par}
	{\Large Jesus Sanchez Granado \par}
	{\Large Victoria Gnatiuk Romaniuk \par}
	\vspace{2cm}
	{\textbf\Large Tutores \par}
	{\Large Simon Pickin \par}
	{\Large Manuel Montenegro Montes \par}
	
\end{titlepage}

%Indice
\tableofcontents
\newpage

\section{Resumen}
Este trabajo de fin de grado es una continuación del trabajo de fin de grado de David Jiménez del Rey en el que se fundamentan las bases de nuestro proyecto. 
Esta es la cuarta parte de un proyecto de desarrollo de una comunidad web de Aprendizaje-Servicio.\\
En este proyecto abarca las fases de analisis, diseño y desarrollo de una aplicación web capaz de gestionar las demandas de servicios sociales de las entidades y las ofertas de servicio de los profesores de las universidades.
Este tipo de asociación entre entidades y centros educativos es una metodología de aprendizaje innovadora llamada Aprendizaje-Servicio o APS.\\
El APS consiste en combinar los conocimientos obtenidos en los centros educativos con los servicios proporcionados por a la comunidad.\\
Aunque una de las ventajas de estos proyectos es la obtención de competencias profesionales, en realidad su principal objetivo es la reflexión, el pensamiento crítico y la responsabilidad social de los alumnos.\\
REVISAR-->
El principal problema de una plataforma orientada para el APS es casar las demandas de las entidades con las ofertas de los profesores. Las entidades suelen ser muy genéricas a la hora de definir requisitos para un proyecto de este estilo y los profesores suelen ofrecer conocimientos más técnicos. Nuestro reto consistía en que la aplicación fuera capaz de casar estas ofertas y demandas y convertirlas en partenariados para posteriormente poder desarrollar un proyecto con dichas características.
TO DO (Describir todo lo que se ha hecho).
\newpage

%Resumen en ingles
\section{Abstract}
\newpage
\section{Introducción}

%TO DO
\subsection{Pruebas de robustez}
Lo primero que hicimos al ponernos a trabajar en el TFG fue realizar una prueba de robustez a la aplicación web de David desplegada en Herouku.\\
Al hacerlo identificamos varios problemas, algunos más graves que otros. Todos estos problemas los fuimos anotando para poder arreglarlos posteriormente en la etapa de desarrollo.\\
Uno de los problemas más graves que encontramos fue el bloqueo de la página web causada por varios acontecimientos como la subida de un fichero a un partenariado o proyecto existentes, la subida y posterior eliminación de una foto de perfil o la subida de una imagen a una iniciativa. Nos percatamos de que todos estos problemas tienen en común la tabla de base de datos llamada "uploads". Esto lo tendremos en cuenta a la hora de corregir los problemas 
posteriormente.

%TO DO
\subsection{Diseño de una nueva base de datos}
Tras analizar la base de datos de David, nos damos cuenta que la estructura de la base de datos será más eficiente siendo una base de datos relacional. Ademas, los profes nos indicaron que el proyecto no estaba bién definido por parte de David, asi que la estructura de la base de datos se tendría que cambiar igualmente.
La base de datos de David, era una base de datos en MongoDB compuesta por 7 documentos. Que representaban el corazón de la aplicación, documentos como: usuarios, proyectos, iniciativas y partenariados.
\begin{figure}[h!]
	\centering
	\includegraphics[scale=0.25]{bdmongo}
	\caption{Base de datos MongoDB}
	\label{fig:universe}
\end{figure}

Tras un exhausto análisis de las necesidades reales de la aplicación, nuestra base de datos en SQL acabó con 43 tablas las cuales, se pueden observar en el siguiente diagrama de entidad relación.
\begin{figure}[h]
	\centering
	\includegraphics[scale=0.15]{bdsql}
	\caption{Base de datos SQL}
	\label{fig:universe}
\end{figure}
\begin{figure}[h]
	\centering
	\includegraphics[scale=0.35]{anuncio}
	\caption{Bloque ``Anuncios de servicio''}
	\label{fig:universe}
\end{figure}
Las tablas que se encuentran en la esquina superior izquierda son tablas relacionadas con lo que David llamaba Iniciativa, que ahora se divide en dos tablas distintas. Una es la demanda de servicio, aquella iniciativa que ha sido creada por una entidad y la cual tiene como propósito solicitar un servicio.
La otra tabla es la oferta de servicio, que es aquella iniciativa que ha sido propuesta por un profesor y que tiene como objetivo ofrecer sus conocimientos. 
Después tenemos una tabla llamada "iniciativa" que es aquella iniciativa que ha sido propuesta por un alumno, esta es un tipo especial de iniciativa que tiene que ser acogida por una entidad para poder aparecer en el aplicativo.
Cuando una oferta y una demanda son casadas y sus respectivos creadores están de acuerdo en cooperar, se crea el partenariado.
\begin{figure}[h]
	\centering
	\includegraphics[scale=0.4]{colaboracion}
	\caption{Bloque ``Colaboración''}
	\label{fig:universe}
\end{figure}
Las tablas relacionadas con el partenariado las encontramos en la esquina inferior derecha. 
En esta parte del esquema podemos observar las tablas que se generar a partir de una colaboración. Tanto el partenariado como el proyecto son especificaciones de una tabla llamada colaboración, aunque el proyecto no puede existir sin que antes se haya creado un partenariado.
\begin{figure}[h]
	\centering
	\includegraphics[scale=0.4]{usuarios}
	\caption{Bloque ``Usuarios''}
	\label{fig:universe}
\end{figure}
En el bloque de abajo a la izquierda nos encontramos con todas las tablas relacionadas con el usuario. Debido a que la plataforma permite el acceso con cuentas internas de las universidades, hemos tenido que duplicar las tablas que recogen los datos de los estudiantes y los profesores. Las tablas de los internos pertenecerían a los usuarios que entran a la plataforma con sus cuentas universitarias y los usuarios externos son aquellos que pertenecen a una universidad que no colabora con la plataforma.
\begin{figure}[p]
	\centering
	\includegraphics[scale=0.4]{comunicacion}
	\caption{Bloque ``Comunicación''}
	\label{fig:universe}
\end{figure}
En el bloque de arriba a la derecha se encuentran las tablas más independientes, aquellas que están relacionadas con la comunicación.
  
\subsection{Objetivos}
Realizamos nuestro TFG en base al trabajo de David Jiménez.
Nuestros principales objetivos fueron rediseñar la aplicación, para que en el futuro se pueda adaptar a un entorno de ejecución real, e implementar una funcionalidad esencial que es el emparejamiento de ofertas y demandas.
A continuación, se detallan estas tareas:
\begin{itemize}
\item \textbf{Pruebas de robustez:} se realizaron pruebas de depuración al código de David. Se observó que el sistema era muy robusto y estable, pero a pesar de ello se detectaron unos pocos fallos menores que fueron arreglados posteriormente.
\item \textbf{Diseño del modelo de dominio:} se observó que el diseño de David no estaba preparado para un entorno de ejecución real y por ello se volvió a rediseñar la aplicación teniendo en cuenta su futura integración en un sistema real. Este diseño fue reflejado en un modelo de dominio que permitió reflejar las entidades, los atributos, las relaciones y las restricciones que gobiernan el ámbito de la aplicación.
\item \textbf{Diseño de la BD:} David había cambiado la BD de MySQL a MongoDb alegando que así podría tener acceso a más documentación. Junto con nuestros tutores, consideramos que la estructura de la aplicación se adapta mejor a una BD relacional y por ello se ha rediseñado la base de datos del sistema a una base de datos SQL.
\item \textbf{Creación de un DAO:} usando la librería knex de Node.js se ha implementado el patrón DAO para ofrecer una interfaz común entre la aplicación y la base de datos, separando de esta manera la lógica de acceso a datos de la lógica de negocio.
\item \textbf{Implementación de un sistema de emparejamiento:} se ha implementado la lógica de emparejamiento de las ofertas de servicio, ofertadas por los profesores, y las demandas de servicio, solicitadas por las entidades beneficiarias resolviendo de esta manera el principal problema de esta aplicación.
\end{itemize}

\subsection{Plan de investigación}
\subsubsection{Fase 0: Documentación sobre el ApS}
Principalmente durante la fase inicial del proyecto, se ha realizado un esfuerzo por comprender el funcionamiento del ApS, para poder entender mejor el diseño de la aplicación y juzgar si había lugares en los que era necesario un enfoque distinto al que se les había otorgado. 
\subsubsection{Fase 1: Revisión y evolución del proyecto}
Durante esta primera fase se ha llevado a cabo un estudio de la memoria proporcionada por David Jiménez en su entrega del TFG, así como de la aplicación tanto en local como en servidor para detectar posibles bugs, de los cuales se hablará en la sección ``Prueba de Robustez''.
Se realiza también un análisis de la base de datos y de las tecnologías escogidas para su realización.
\subsubsection{Fase 2: Implementación y Documentación}
Lo primero que se ha hecho ha sido arreglar diversos bugs de menor o mayor importancia que han sido encontrados durante la fase de prueba de robustez, tras lo que se ha procedido a un rediseño de la BD en SQL, dado que la gran mayoría de los datos guardados por la aplicación son estructurados, nos ha parecido una tecnología más adecuada que la de MongoDB.\\
Durante este proceso ha surgido la necesidad de rediseñar algunas partes importantes de la base de datos, como la división de la tabla de ``Iniciativa'' en dos partes. Por un lado, tendríamos la ``oferta de servicio'' propuesta por un profesor y por otro lado la ``demanda de servicio''   solicitada por una entidad. Esta separación ha sido necesaria porque las necesidades de una organización difieren bastante de lo que puede considerar adecuado ofrecer un profesor y por tanto no podemos almacenar todos estos requerimientos en una única tabla.\\
Tanto esta fase de diseño como la posterior implementación del sistema de emparejamiento de ofertas y demandas han constituido uno de los pilares de este proyecto.
\bibliographystyle{plain}
\bibliography{references}
\end{document}