\documentclass{article}
\usepackage[utf8]{inputenc}

\usepackage{graphicx}
\usepackage[dvipsnames]{color}
\usepackage[hidelinks]{hyperref}
\usepackage[numbers]{natbib}
%pPara poder modificar los margenes
\usepackage{vmargin}
%Para usar el español
\usepackage[spanish]{babel}

\begin{document}
%Portada
\setpapersize{A4}
\begin{titlepage}
	\centering
	{\includegraphics[width=0.8\textwidth]{logo}\par}
	\vspace{1cm}
	{\Large Facultad de Ingeniería Informática \par}
	\vspace{3cm}
	{\scshape\Huge Aplicación web de soporte al Aprendizaje-Servicio \par}
	\vspace{5cm}
	{\textbf\Large Autores \par}
	{\Large Daniela Nicoleta \par}
	{\Large Jesus Sanchez Granado \par}
	{\Large Victoria Gnatiuk Romaniuk \par}
	\vspace{2cm}
	{\textbf\Large Tutores \par}
	{\Large Simon Pickin \par}
	{\Large Manuel Montenegro Montes \par}
	
\end{titlepage}

%Indice
\tableofcontents
\newpage

\section{Resumen}
Este trabajo de fin de grado es una continuación del trabajo de fin de grado de David Jiménez del Rey en el que se fundamentan las bases de nuestro proyecto. 
Esta es la cuarta parte de un proyecto de desarrollo de una comunidad web de Aprendizaje-Servicio.\\
En este proyecto abarca las fases de analisis, diseño y desarrollo de una aplicación web capaz de gestionar las demandas de servicios sociales de las entidades y las ofertas de servicio de los profesores de las universidades.
Este tipo de asociación entre entidades y centros educativos es una metodología de aprendizaje innovadora llamada Aprendizaje-Servicio o APS.\\
El APS consiste en combinar los conocimientos obtenidos en los centros educativos con los servicios proporcionados por a la comunidad.\\
Aunque una de las ventajas de estos proyectos es la obtención de competencias profesionales, en realidad su principal objetivo es la reflexión, el pensamiento crítico y la responsabilidad social de los alumnos.\\
REVISAR-->
El principal problema de una plataforma orientada para el APS es casar las demandas de las entidades con las ofertas de los profesores. Las entidades suelen ser muy genéricas a la hora de definir requisitos para un proyecto de este estilo y los profesores suelen ofrecer conocimientos más técnicos. Nuestro reto consistía en que la aplicación fuera capaz de casar estas ofertas y demandas y convertirlas en partenariados para posteriormente poder desarrollar un proyecto con dichas características.
TO DO (Describir todo lo que se ha hecho).
\newpage

%Resumen en ingles
\section{Abstract}
\newpage
\section{Introducción}

%TO DO
\subsection{Pruebas de robustez}
Lo primero que hicimos al ponernos a trabajar en el TFG fue realizar una prueba de robustez a la aplicación web de David desplegada en Herouku.\\
Al hacerlo identificamos varios problemas, algunos más graves que otros. Todos estos problemas los fuimos anotando para poder arreglarlos posteriormente en la etapa de desarrollo.\\
Uno de los problemas más graves que encontramos fue el bloqueo de la página web causada por varios acontecimientos como la subida de un fichero a un partenariado o proyecto existentes, la subida y posterior eliminación de una foto de perfil o la subida de una imagen a una iniciativa. Nos percatamos de que todos estos problemas tienen en común la tabla de base de datos llamada "uploads". Esto lo tendremos en cuenta a la hora de corregir los problemas 
posteriormente.

%TO DO
\subsection{Diseño de la base de datos}
La base de datos se va a cambiar a una base de datos SQL ya que nos damos cuenta que la estructura de la base de datos es una base de datos relacional.
\bibliographystyle{plain}
\bibliography{references}
\end{document}