\documentclass{report}
\usepackage[utf8]{inputenc}

\usepackage{graphicx}
\usepackage[dvipsnames]{color}
\usepackage[hidelinks]{hyperref}
\usepackage[numbers]{natbib}

\begin{document}
\begin{titlepage}
	\centering
	{\includegraphics[width=0.8\textwidth]{logo}\par}
	\vspace{1cm}
	{\Large Facultad de Ingeniería Informática \par}
	\vspace{3cm}
	{\scshape\Huge Trabajo de  fin de grado de aplicación web APs \par}
	\vspace{3cm}
	{\Large Daniela Nicoleta \par}
	{\Large Jesus Sanchez Granado \par}
	{\Large Victoria Gnatiuk Romaniuk \par}
\end{titlepage}
%Sección de resumen
\begin{abstract}
\end{abstract}
%Otras secciones
\section{Introducción}
\section{Pruebas de robustez}
Lo primero que hicimos al ponernos a trabajar en el TFG fue realizar una prueba de robustez a la aplicación web de David desplegada en Herouku.\\
Al hacerlo identificamos varios problemas, algunos más graves que otros. Todos estos problemas los fuimos anotando para poder arreglarlos posteriormente en la etapa de desarrollo.\\
Uno de los problemas más graves que encontramos fue el bloqueo de la página web causada por varios acontecimientos como la subida de un fichero a un partenariado o proyecto existentes, la subida y posterior eliminación de una foto de perfil o la subida de una imagen a una iniciativa. Nos percatamos de que todos estos problemas tienen en común la tabla de base de datos llamada "uploads". Esto lo tendremos en cuenta a la hora de corregir los problemas 
posteriormente.
\section{Diseño de la base de datos}
La base de datos se va a cambiar a una base de datos SQL ya que nos damos cuenta que la estructura de la base de datos es una base de datos relacional.
\bibliographystyle{plain}
\bibliography{references}
\end{document}